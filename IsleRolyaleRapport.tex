\documentclass[a4paper]{report}
\usepackage{tikz}
\usepackage[margin=2.5cm]{geometry}
\usepackage{hyperref}
\usepackage{graphicx}
\graphicspath{{figures/}{anotherFigureDirectory/}}
\graphicspath{ {./images/} }
\usepackage{listings}
\usepackage{wrapfig}
\usepackage{color}
\definecolor{bluekeywords}{rgb}{0.13,0.13,1}
\definecolor{greencomments}{rgb}{0,0.5,0}
\definecolor{turqusnumbers}{rgb}{0.17,0.57,0.69}
\definecolor{redstrings}{rgb}{0.5,0,0}
\definecolor{gray}{rgb}{0.13,0.13,0.13}
\lstdefinelanguage{FSharp}
                {morekeywords={let, new, match, with, rec, open, module,
                namespace, type, of, member, and, for, in, do, begin, end, fun,
                function, try, mutable, if, then, else},
                keywordstyle=\color{bluekeywords},
                sensitive=false,
                numbers=left,  % where to put the line-numbers;(none, left, right)
                numberstyle=\tiny\color{gray},
                morecomment=[l][\color{greencomments}]{///},
                morecomment=[l][\color{greencomments}]{//},
                morecomment=[s][\color{greencomments}]{{(*}{*)}},
                morestring=[b]",
                showstringspaces=false,
                stringstyle=\color{redstrings}
                }

\title{PoP - Ugeopgave 10}
\author{Christoffer, Inge og Pernille}
\date{\today}

\begin{document}
\maketitle

\section*{To-Do liste}

\begin{enumerate}

\item Skriv kode for "tick"-funktionen, der styrer tiden for spillet og implementer ticks funktionen i de øvrige objekter (se Jons kommentarer i koden).

This.tick funktionerne skal relateres til, at ulven tjekker, om der er et Moose i et nabofelt, hvis ja: skal ulven rykkes til Moose'ets felt, Moose'et fjernes og ulvens sult resetes.

Ulvens sult skal relateres til tick's, så hver gang der er gået et tick, kaldes metoden for ulvens sult (en funktion der allerede er givet), som tæller sulten 1 værdi ned.

For hvert tick, skal en ulvs sult tjekkes, hvis sulten er nået 0, skal ulven dø -> fjernes fra miljøet.
\item Hvordan spiser en ulv en elg?
\item Hvordan indsættes ungerne?
\item Hvad betyder rnd.Next? Særligt i tilknytning til reproduction.
\item n
\item n
\item n
\item n

\end{enumerate}

\newpage

\subsection*{Introduction}

\subsection*{Problemanalyse \& design}

\subsection*{Programbeskrivelse}

\subsection*{Test (afprøvning)}

\subsection*{Eksperiment}

\subsection*{Konklussion}

\newpage
\section*{Bilag}

\subsection*{Brugervejledning}

\subsection*{Programtekst}

\end{document}